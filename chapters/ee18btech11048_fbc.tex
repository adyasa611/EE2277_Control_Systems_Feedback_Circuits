\begin{enumerate}[label=\thesubsection.\arabic*.,ref=\thesubsection.\theenumi]
\numberwithin{equation}{enumi}

\item
Figure \ref{fig:original_circuit} shows a feedback transconductance
amplifier implemented using an op amp with open-loop gain $\mu$
, a very large input resistance, and an output resistance $r_{o}$.
The output current $I_{o}$ that is delivered to the load resistance $R_{L}$
is sensed by the feedback network composed of the three
resistances $R_{M}$, $R_{1}$, and $R_{2}$, and a proportional voltage$V_{f}$
is fed back to the negative-input terminal of the op amp.\\

\begin{figure}[!ht]
	\begin{center}
		\resizebox{\columnwidth}{!}{\input{./figs/ee18btech11048_fbc1.tex}}
	\end{center}
\caption{1 Original Circuit}
\label{fig:original_circuit}
\end{figure}\\


Find G,H and T. If the loop gain is large, find an approximate expression for T
and state precisely the condition for which this applies.\\

\solution
The parameters given are shown in the TABLE.\ref{table: Table1}:1
\begin{table}[!ht]
\centering
\input{./tables/ee18btech11048_table1.tex}
\caption{1}
\label{table: Table1}
\end{table}
The equivalent circuit of the amplifier is in fig.\ref{fig:ss_circuit}:2

\begin{figure}[!ht]
	\begin{center}
		\resizebox{\columnwidth}{!}{\input{./figs/ee18btech11048_fbc2.tex}}
	\end{center}
\caption{2 Equivalent Circuit}
\label{fig:ss_circuit}
\end{figure}\\
\item
Calculating G\\
\solution
\begin{align}
G &= \frac{I_{o}}{V_{i}} \label{eq:G}\\
\text{From fig \ref{fig:ss_circuit}:2}\\
\implies G&= \mu
\end{align}
\item
Calculating H\\
\solution
\begin{align}
H &= \frac{V_{f}}{I_{o}} \label{eq:H}
\end{align}
From fig \ref{fig:ss_circuit}:2\\
\begin{align}
V_{f}&=R_{1}I_{o}\frac{R_M}{R_M+R_1+R_2}\\
\implies
H &= \frac{R_1R_M}{R_1+R_2+R_M}
\end{align}
\item
Calculating T\\
\solution
\begin{align}
T &= \frac{G}{1+GH} \label{eq:T}
\end{align}
From fig \ref{fig:ss_circuit}:2
\begin{align}
T&= \frac{\mu \brak{R_1+R_2+R_M}}{R_1+R_2+R_M+ \mu R_1R_M}
\end{align}


\begin{table}[!ht]
\centering
\input{./tables/ee18btech11048_tables2.tex}
\caption{1}
\label{table: Input_Table}
\end{table}


\item When Loop Gain is large\\
\solution
\begin{align}
GH &\gg 1,
 \\
T &\approx \frac{1}{H}  = \frac{R_1+R_2+R_M}{R_1R_M} 
\end{align}
This is the key to designing a successful feedback system; if we can guarantee that $GH \gg 1$ for the frequencies that we are interested in, then the closed-loop gain will not be dependent on the details of the plant gain G. This is very useful, since in some cases the feedback function H can be implemented with a simple resistive divider, which can be cheap and accurate.

\end{enumerate}
